Some of the major hypotheses found in SE literature which could be studied in this project are:
\begin{itemize}
    \item Moving from traditional to agile practices could reduce the effort and cost to develop a software. Agile practices promises lower defective rates~\cite{moniruzzaman2013comparative}.
    \item High priority problems should resolved faster than lower priority problems~\cite{moniruzzaman2013comparative}.
    \item Agile practices promises lower resolution time for issues/bugs~\cite{mockus2002two}.
    \item Agile practices are more suitable for smaller team than larger team projects~\cite{gandomani2013obstacles}.
    \item Morale and Productivity of developers and project completion  increases  in  agile  methodology~\cite{dybaa2008empirical}.
    \item  Pull requests are more likely to result in integration testing failures than  push  commits~\cite{vasilescu2014continuous}.
    \item The number of unique authors who have contributed less than 5\% of the changes could represent defective areas~\cite{mcintosh2016empirical}. That is why, hero (in which majority of work is done by  only 1 programmer) projects are not ideal.
    \item Changes produced by novice developers are more likely to introduce defects than those produced by subject matter experts~\cite{mcintosh2016empirical}.
    \item Software Development which adopts Continuous Integration have better software quality~\cite{vasilescu2014continuous}.
    \item Usually a project adopts any continuous integration tool in a year~\cite{hilton2016usage}.
    \item Project leaders tend to have lesser ties with other developers. Furthermore, novice developers first tend to communicate with fellow novice developers first, and gain experience and reputation. Once they get experienced they tend to bond with other fellow experienced developers more~\cite{shen2011connects}.
    \item Core developer bugs, i.e., those reported internally, by developers, and user bugs, i.e., those reported externally, by users could identify quality of a project~\cite{vasilescu2015quality}.
    \item Adding more developers into a project in between increases the ramp up time for them to become productive~\cite{brooks1975mythical}.
\end{itemize}