\paragraph{Anti-patterns}

Anti-patterns are poor solutions to recurring design problems. They occur in object-oriented systems when developers unwillingly introduce them while designing and implementing the classes of their systems~\cite{fowler99}. In addition to anecdotal evidence, in the past decade anti-patterns have been empirical shown to have an detrimental impact on software quality~\cite{Ab11, Ar11, Ch10, Kh12, Li07}. There is also very strong evidence that the number of anti-patterns in software systems increases over time and only in few cases are they removed through refactoring operations~\cite{Ar11,Ch10}. Researchers have called for recommendation systems supporting the software engineer in (i) identifying anti-patterns and (ii) designing and applying a refactoring solution to remove them~\cite{Ba98}. Among the 30+ anti-patterns defined in the literature~\cite{fowler99,Ky05}, only for a subset of them we have
approaches and tools for their automatic identification. A survey of literature suggest that most widely used tools such as DECOR~\cite{decor} use a set of rules, called “rule card”, describing the intrinsic characteristics of a module/class affected by anti-patterns. Beside DECOR, other approaches also rely on rules generated by metrics to identify anti-patterns in source code. For example,
Marinescu~\cite{Ma04} presented a detection strategy able to identify anti-patterns by generating rule derived from deviations from good design principles. Fowler \cite{fowler99} and Brown
et al.~\cite{Br98} defined more than 30 anti-patterns. However, in the previous year, researchers concentrate their attention only on a small subset of anti-patterns defined in the literature. Palomba et al.~\cite{Pa15} assert that this limitation is due the difficulty in generating rules for identification of code smells. Methods such as CrossTREE may be particularly useful in these cases. In fact, Krishna et al.~\cite{krishna17a} have already shown that CrossTREE can be used to reason effectively about anti-pattern to assist reorganization efforts to address them. It must be noted that by augmenting CrossTREE with developer insights and knowledge combination methods such as PSO, it may be possible to further assist addressing anti-patern.