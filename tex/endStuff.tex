%%%%%%%%%%%%%%%%%%%%%%%%%%%%%%%%
\head{7. BROADER IMPACT AND EDUCATION:}
\paragraph{7.1. Benefits to Society at Large}
This proposal focuses on issues of
tremendous economic importance -- the creation of better quality software.
As a result, this work will increase America's ability for industrial and academic innovators to conduct
more scientific studies via computational means.  
 

\paragraph{7.2. Integration of Teaching and Research}
Much of the research in this project will also be integrated into a
classroom environment.  The PI teaches senior-level and graduate-level
SE, and software analytics   classes (and in those data mining
classes, all the case studies come from software engineering). All of
the technology developed in this proposal will become study
material for those subjects.  Replication studies are especially ripe
for classroom projects.  Also, through our publications and conference
work, we would publicize these tools as widely as possible with the
intent of supporting a broader community working with this approach.

\paragraph{7.3 Participation of Underrepresented
  Groups} The PI will continue their established tradition of
graduating research students for historically under-represented
groups.  PI Menzies' last two Ph.D. students were an African-American
women and a gentleman from an economically disadvantaged region
in central Pennsylvania.  Also, each summer, the PI's department runs an NSF-funded REU
 (research experience for undergraduates)
on the ``Science of Software''.
 At this
program, places are reserved for students from traditionally
under-represented areas (e.g. economically challenged regions of the
state of North Carolina) and/or students from universities
lacking advanced research  facilities.
Some of the simpler data mining concepts for this proposal would be suitable for lectures and REU projects.



%\input{f/under}
\paragraph{7.4. Dissemination of Knowledge}
 The PI has an extensive history of publishing at senior SE venues and so, it should be expected
that the results of this work will be widely visible.

Also, as mentioned above, the PI has a long history of publishing papers along with reproduction packages holding the data and the scripts required to
replicated the papers' results. All the our methods used here will be based around software tools in widespread   use (Github, Travis CI, etc).  We will release all our tools via open source licenses so our results can be readily applicable to researcher or developers using   Github, etc.
Although some of the data used in this study comes from private in-house projects,
the majority of our data comes open-source projects which we share with the community.

One issue here will be that it might be  too expensive
to let other groups download our  ingested data:
\bi
\item While we can certainly store our raw data on  S3,  Amazon   charges 4 cents per gigabyte download. Assuming our 5TB of data is downloaded 20 times a month for 3 years, that would cost \newline (\$0.05*1024*5*20)*36=\$184,320 $\approx$ 37\% of the grant (i.e. too much).
\item On the other hand, it is possible that our ingested data will occupy only a small fraction of the raw data-- in which case we can give free access
to our data to all
research groups.
\ei
Regardless of the above two points, we can certainly share all our scripts and bad smell detectors. These would allow other groups to replicated our results
after they conducted their own downloads.
 
For more on this point, see our Data Management Plan.

 


\head{8. PRIOR RESULTS:}
This proposal is the next logical step in the PI's work on analytics.
{PI Menzies} has worked extensively in that arena.
\textbf{Dr. Tim Menzies} has several prior NSF  grants relevant to this work. All those grants
used data miners to predict properties of software artifacts. 
Two  older grant were:
\bi
\item
\underline{(a)}~CCF-0810879,2008 to 2012, \$350,000;
\underline{(b)}~``CPA-SEL: Automated Quality Prediction: Exploiting Knowledge of the Business Case''.
 and exper(c)that 
\item \underline{(a)}~CCF-1017330 (2011 to 2015,  \$350,000),
\underline{(b)}~``SHF: Small: Collaborative Research: Better Comprehension of Software Engineering Data''. 
\ei
\underline{(c)}~The {\bf intellectual merit} of these grants were to define the data, evaluation methods, challenge problems and baseline results for this style of data mining.  The {\bf broader impact} of those grants was to  that inspired literately hundreds of subsequent papers,
 by other researchers.
These two grants generated \underline{(d)}~31 publications~\cite{%
jiang08a,%
me09n,%
me09i,%
me09b,%
me10d,%
Bavota2010,%
Bavota2012a,%
Bavota2013,%
Bavota2012b,%
Haiduc2010a,%
Haiduc2013a,%
Haiduc2012a,%
Haiduc2013b,%
Marcus2010b,%
Ohlemacher2011b,%
Scanniello2013,%
Scanniello2011,%
me11m,%
peters12,%
Me13,%
me13a,peters12a} and 
partially supported ten masters students and five Ph.D. students. \udnerline{(e)}~Data from that work is now housed at the publicly accessible repository http://tiny.cc/seacraft and lessons learned from that work was widely disseminated in the book ``Sharing Data and Models in SE''~\cite{Menzies:2014:SDM:2930830}. \underline{(f)} N/A.  



More recently, PI Menzies worked on \underline{(a)}~CCF-1302216, 2013-2107, \$271,553;   \underline{(b)}~``SHF: Medium: Collaborative: Transfer Learning in Software Engineering''. 
 \underline{(c)}~The \underline{{\em intellectual merit}} of that work was to
define novel methods for sharing data, many of which were the precursor to the methods of this proposal.  That work generated the publications  \underline{(d)}~\cite{krishna2018bellwethers,PetersML15,krishna16,he13,Me17,fu2016tuning,krishna2017learning,nam2017heterogeneous}, that has been privatized
yet still  capable of building effective
models
for prediction and planning purposes.
The \underline{{\em broader impact}} of that work was to
enable a new kind of open science-- one where all data is routinely shared and is capable of building effective models no matter if it is obfuscated for security proposes.
The methods of this project, while targeted at software engineering, could also be applied to any other data intensive field.   
 \underline{(e)}~Data from that work is now housed at the publicly accessible repository http://tiny.cc/seacraft. That work  funded two Ph.D.s at NCSU. \underline{(f)}
N/A.  

There are other grants that could be listed here but as we understand recent changes to NSF grant guidelines, the above are all that is allowed/required.