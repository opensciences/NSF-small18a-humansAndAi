%%%%%%%%%%%%%%%%%%%%%%%%%%%%%%%%
\head{7. BROADER IMPACT AND EDUCATION:}
\paragraph{7.1. Benefits to Society at Large}
This proposal focuses on issues of
tremendous economic importance -- the creation of better quality software.
As a result, this work will increase America's ability for industrial and academic innovators to conduct
more scientific studies via computational means.  
 

\paragraph{7.2. Integration of Teaching and Research}
Much of the research in this project will also be integrated into a
classroom environment.  The PI teaches senior-level and graduate-level
SE, and data mining classes (and in those data mining
classes, all the case studies come from software engineering). All of
the technology developed in this proposal will become study
material for those subjects.  Replication studies are especially ripe
for classroom projects.  Also, through our publications and conference
work, we would publicize these tools as widely as possible with the
intent of supporting a broader community working with this approach.

\paragraph{7.3 Participation of Underrepresented
  Groups} The PI will continue their established tradition of
graduating research students for historically under-represented
groups.  PI Menzies' last two Ph.D. students were an African-American
women and a gentleman from an economically disadvantaged region
in central Pennsylvania.  Also, each summer, the PI's department runs an NSF-funded REU
 (research experience for undergraduates)
on the ``Science of Software''.
 At this
program, places are reserved for students from traditionally
under-represented areas (e.g. economically challenged regions of the
state of North Carolina) and/or students from universities
lacking advanced research  facilities.
Some of the simpler data mining concepts for this proposal would be suitable for lectures and REU projects.



%\input{f/under}
\paragraph{7.4. Dissemination of Knowledge}
Recall our introductory notes-- one of significant features of this work is its
immediate industrial applicability.
All the our methods used here will be based around software tools in widespread industrial use (Jira,Github, Travis CI, etc).  We will release all our tools via open source licenses so our results can be readily applicable to researcher or developers using Jira, Github, etc.
Although some of the data used in this study comes from private in-house projects,
the majority of our data comes open-source projects which we share with the community.
Further, the PI has an extensive history of publishing at senior SE venues and so, it should be expected
that the results of this work will be widely visible.


\head{8. PRIOR RESULTS:}
This proposal is the next logical step in the PI's work on analytics.
{\bf PI Menzies} has worked extensively in that arena.
s
In
\emph{``Automatic
Quality Assessment: Exploiting Knowledge of the Business Case''} (CCF-0810879, \$350,000, 08/08-06/11),
{\bf PI Menzies} (with Bojan Cukic, WVU) built   generated many papers~\cite{jiang08a,me09n,me09i,me09b,me10d}
as well as one that is  was the third most-cited
paper in period 2009 to 2014 in the Journal of Empirical Software
Engineering.

In ``Better Comprehension of Software Engineering Data'' (CCF-1017330, \$500,000, 08/10-08/14).
{\bf PI Menzies} (with Andrian Marcus, Wayne State),
explored various learning strategies (kernel estimation
methods, active learning tools, privatization methods) to
support defect and effort estimation. That work generated many papers~\cite{Bavota2010,Bavota2012a,Bavota2013,Bavota2012b,Haiduc2010a, Haiduc2013a, Haiduc2012a, Haiduc2013b, Marcus2010b, Ohlemacher2011b, Scanniello2013, Scanniello2011,me11m, peters12,Me13,me13a,peters12a}.
In terms \emph{broader impact}, that work graduated four masters and four Ph.D. students
and
spawned a small but active ``local learning'' research sub-community in SE (PI Menzies showed that
such ``local learners'' find better models
by first inferring local contexts, then building one model per context~\cite{Me13,me11m}).

Currently, PI Menzies is concluding
``Transfer Learning in Software Engineering'' (CCF-1302216, \$1,100,000, 08/13-07/18)
where {\bf PI Menzies} (with Lucas Layman from Franhoufer Institute) is
building a toolkit to explore  methods for moving knowledge
learned from one software project to another.  That work  has generated
papers at ICSE'15~\cite{PetersML15}, ASE'15~\cite{krishna16}  and ESEM'13~\cite{he13} as well as the EMSE journal~\cite{Me17}, two IST journal papers~\cite{fu2016tuning,krishna2017learning},
one TSE paper~\cite{nam2017heterogeneous}. Other papers are under review at JSS~\cite{rees2017better}
and TSE~\cite{krishna2017simpler}. A surprising and very useful outcome of that work
is the discovery of ``bellwethers'',  a simple transfer learner.

Lastly, PI Menzies has just started ``Scalable Holistic Autotuning for Software Analytics''
(CCF-1703487, \$898,349, 7/1/17-6/30/21) which is work with Co-PI Xipeng Shen
on using compiler technology to optimize very slow software analytics workflows. 
At the time of this writing, there is nothing to report on this grant.
